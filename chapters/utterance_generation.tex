\section{Utterance Generation}\label{sec:STDT}

This section describes our method to generate training examples. Our goal is to
create sentences that are structured but similar to unstructured natural
language. The structure is given by the CFG and the templates crafted carefully
so the resulting sentences still satisfy the syntactic rules of natural
language. We call these sentences structured task descriptions. Each of these
descriptions has a natural language equivalent. \cref{fig:natural} shows a
(structured task description, natural language sentence) pair. The task
description will be used as the running example in this section.

\insertPic{natural}{0.75}{A structured task description (bottom) and its natural
  language counterpart (top). The intermediate step is included for explanation
  purposes, in order to make the same parts in the two sentences more
  identifiable.}

The templates are handcrafted by expanding the algorithm rule (\cref{fig:arule})
with extra tokens based on natural language utterances describing tasks.
Templates are created for specific algorithmic patterns in order to get a result
close to natural language. This is why in each template the \verb|<indicator>|
field is fixed -- its value is predetermined by the indicator rule
(\cref{fig:irule}). \cref{fig:template} shows a valid template used during the
dataset generation.

\insertPic{arule}{0.75}{The algorithm rule. The templates are created by
  expanding this rule with additional tokens to make it similar to natural
  language.}

\insertPic{irule}{0.75}{The indicator rule. Each possible value maps to one of
  the algorithmic patterns described in \cref{sec:pattern}.}

\insertPic{template}{0.64}{A template to generate structured task descriptions
  of the summation algorithmic pattern. It has the same fields as the
  nonterminals of the algorithm rule in \cref{fig:arule} as it was created by
  expanding that rule with additional tokens (except that it has a single
  condition).}

The templates give structure to the task descriptions. Each field in a template
can be filled in multiple ways (\cref{fig:task}). We can obtain many examples
from a template so we can compile a diverse dataset by using a relatively small
number of templates. The \texttt{<conditions>} rule shown in \cref{fig:crule}
also helps to make our dataset more diverse as it can generate a wide variety of
conditions.

\insertPic{task}{0.59}{Creating a structured task description from a template by
  filling in its fields. The value of each field is randomly selected from
  its list of possible values. The figure has two possible values in each list
  for demonstration.}

\insertPic{crule}{0.75}{The condition rule can generate different conditions in
  the templates.}

The Python snippets are generated from the filled templates. First the
algorithmic pattern and its code structure is selected based on the
\texttt{<indicator>} field. Then the rest of the fields are used to substitute
the corresponding components in the code structure to yield the code snippet.
\cref{fig:connect} shows an example of code snippet generation from a template.

\insertPic{connect}{0.6}{Given a filled template, one can produce the
  corresponding code snippet by selecting the code structure based on the
  \texttt{<indicator>} field and filling its components.}