\section{Algorithmic Patterns}
\label{sec:pattern}

Generic algorithmic patterns\cite{progT, progT1} are applied to
solve frequently occurring programming problems, like counting
or summing elements of a collection\cite{progSolve}. The problems can appear in different forms
but their substance is the same, so we can use analogies with the algorithmic
patterns to solve them. This section describes the chosen algorithmic patterns
and their corresponding Python snippets.

Algorithmic patterns come from the area of analogous programming\cite{progT}, a
formal software development method where programs are composed from provably
correct parts.

We are using the following algorithmic patterns, detailed below:

\begin{enumerate}[noitemsep]
  \item Summation
  \item Count
  \item Maximum search
  \item Selection
  \item Decision
  \item Assortment
\end{enumerate}

\subsection{Formalism}\label{sec:def}

In order to discuss the used patterns effectively, formalisms need to be
introduced. These will be used to define algorithmic patterns and they are
relevant in the construction of the CFG rules.

Let us introduce the set \( \mathcal{H} \), a set on which a given task is
defined. For example, if our task is to sum the elements of set
\( \{ 1, 2, 6, 7, 12, 5 \} \), then \( \mathcal{H} = \mathbb{N} \). In this case
\( h \in \mathcal{H} \) is a natural number.

In our tasks the elements are members of a \( collection \) which is a sequence
holding the relevant elements of \( \mathcal{H} \). We can iterate over a
collection. We are using Python 3, so in our concrete case \( collection \) is a
Python list and \( h \in collection \) is an element of the list.

An important thing to consider is that in many cases, \( \mathcal{H} \) contains
compound elements. For example, if our task is to find the tallest student in a
classroom, \( \mathcal{H} \) can contain objects (records) of students with
different attributes. We have to select the relevant property from e.g., name,
height, grades, etc. We denote the set of values of the relevant property by
\( \mathcal{V} \). In the example height is the relevant property, therefore
\( \mathcal{V} = \mathbb{N} \).

We also introduce a mapping function \( f: \mathcal{H} \to \mathcal{V} \) to map
the elements of the \( collection \) to their relevant value. In the example
\( f \) would map students to their heights. In case of primitive types, like
natural numbers, \( \mathcal{H} = \mathcal{V} \) and \( f \) is the identity
function.

Finally, the filtering function \( c: \mathcal{H} \to \mathbb{L}\) maps
\( x \in \mathcal{H} \) to a logical value, based on a condition. The role of
function \( c \) is to determine which elements of the \( collection \) take
part in the calculation.

\subsection{The Algorithmic Patterns}

This section describes the algorithmic patterns we use and provides a generic
Python code snippet for each. Since our dataset contains \emph{(utterance, code)}
pairs, these snippets can be used as templates for the latter. During the
creation of these snippet templates the emphasis was on understandability,
therefore it is possible that some of these templates are not the most
efficient. However changing the templates are relatively easy if efficiency
becomes more important later.

\paragraph{Summation}\label{sec:sum}
is one of the most applicable patterns\cite{Sum}. It is used when items of a
collection need to be summed:
\begin{listing}[H]
\begin{minted}{python}
    r = sum(f(x) for x in collection if c(x) )
\end{minted}
\end{listing}


\paragraph{Count}
is used to count those items of a collection that satisfy a certain condition:
\begin{listing}[H]
\begin{minted}{python}
    r = sum(1 for x in collection if c(x) )
\end{minted}
\end{listing}

\paragraph{Maximum search}
is an algorithmic pattern where our task is to find the maximum element
according to comparison operator \( \leq \):
\begin{listing}[H]
\begin{minted}{python}
    r = max(f(x) for x in collection),
\end{minted}
\end{listing}

where the \( max \) function uses the comparison operator. Although this
formulation describes the algorithm well when the task is to find the maximal
element according to the operator \( \leq \), it falls short when a condition is
introduced. In order to handle such cases, one can extend the algorithm with a
condition. In this case, it is assumed that there is at least one
element satisfying the condition. The new form of the algorithm augmented by
condition \( c \):

\begin{listing}[H]
\begin{minted}{python}
    r = max(f(x) for x in collection if c(x) )
\end{minted}
\end{listing}

\paragraph{Selection}
is used when the task is to find the first element matching a condition. The
precondition is that at least one element satisfies the condition.
\begin{listing}[H]
\begin{minted}{python}
    r = next(f(x) for x in collection if c(x) )
\end{minted}
\end{listing}


\paragraph{Decision}
 determines whether there is an element that satisfies the given condition \( c \). 
\begin{listing}[H]
\begin{minted}{python}
    r = any(c(x) for x in collection)
\end{minted}
\end{listing}

The problem to decide whether all elements meet the given condition \( c \) can
be formulated in similar fashion.

\begin{listing}[H]
\begin{minted}{python}
    r = all(c(x) for x in collection)
\end{minted}
\end{listing}

\paragraph{Assortment}
is used when a set of elements needs to be separated into two subsets: one where
all elements meet a certain condition and another where elements do not meet
this condition.
\begin{listing}[H]
\begin{minted}{python}
    r1 = [x for x in collection if c(x) ]
    r2 = [x for x in collection if not c(x)],
\end{minted}
\end{listing}
where \( r1 \) and \( r2 \) do not contain any identical items.