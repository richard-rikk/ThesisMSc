\section{Details of the model}
TranX is a complex web of technologies working collectively in order to create meaning
representations from natural language descriptions. This section unravels the web by
explaining in detail the more important parts of the model. These parts are the problem
statement, the transition system, and the neural network architecture itself.

\subsection{Problem Statement}

\subsubsection{Semantic Parsing}\label{sec:parsing}

The general task of semantic parsing is a mapping of a natural language query to a
corresponding formal meaning representation. This can be formalized as finding the relation
\(\varphi \subseteq NLQ \times FMR\)\label{form:fmr}, where for a given query \(q\) a 
representation \(r\) corresponds to \(q\) if \(r \in \varphi(q)\). In this case,
\( \varphi(q) \) is a set emphasizing that a query may have multiple correct 
formal representations. One way of evaluating these is by defining a fit
function \( FMR \to \mathbb{R} \) and taking \( r \) with the highest value. 
However, in our case, every element of \( \varphi(q) \) has the same fitness,
meaning, our only consideration is that \(r\) corresponds to \(q\).

For our purposes, a \(q = w_{1} \dots w_{|q|} \in NLQ\) query is a correctly
formulated body of text in a given natural language, expressed as a sequence of words.
An \(r = s_{1} ... s_{|r|} \in FMR\) representation is a sequence of terminal symbols
that is valid in a given formal language.

The relationship captured by \(\varphi\) is semantic by nature. When \((q, r)\) is
an element of \(\varphi\), it is intended to imply that the human-understood
meaning of \(q\) can be equated with the machine-understandable \(r\). 
 
\subsubsection{Intent Parsing}

Intent parsing is a special case of semantic parsing. In intent parsing, the meaning of 
\(q\) is the object of the intention, the action intended to be carried out. The form in 
which this meaning is to be represented after semantic parsing is such that the original 
action may be performed based solely on the representation.

Intent parsing can be used to model the semantic transition that takes place when writing
a computer program. The programmer has an action in mind that is to be performed by the
computer upon running the program. This action's description is already formulated by
the programmer in a natural language before it is implemented in a programming language.

In our case, a natural language description of a programming task (an action to be
performed by a computer) serves as the input, and a program that solves the described
task (performs the required action) is the output.

\subsection{Transition System}

\subsubsection{Transition Process}

A transition system \(\tau\) is a bijective function from an arbitrary set \(IMR\) to \(FMR\),
in our case \(IMR\) is a set of AST constructing action sequences and \(FMR\) is a set of formal
meaning representations. Using the transition system, one can solve the intent parsing task of
finding an \(r \in \varphi(q)\) for a given \(q\) by finding an \(a \in IMR\) for which
\(\tau(a) = r \in \varphi(q)\). In this sense, \(IMR\) is the set of intermediate meaning
representations, and the transition system is the method of conversion between them and the
final outputs, the elements of \(FMR\).

\textbf{\textsc{Decomposition}}\ \ \ The transition system can take several steps to get
from an \(a \in IMR\) to an \(r \in FMR\), and similarly, its inverse \(\tau^{-1}\)
can take these same steps in a backward order to perform the conversion in the other direction.
This essentially means that the transition system is a function composition
\(\tau = \tau_{n} \circ ... \circ \tau_{1}\) and \(\tau^{-1} = \tau_{1}^{-1} \circ \dots 
\circ \tau_{n}^{-1}\), where each \(\tau_{i}\) is bijective.

\textbf{\textsc{Transition Process}}\ \ \ In the case of TranX, a transition system is utilized
for which \(\tau^{-1}\) works similarly to a compiler. \(\tau_{5}^{-1}\) is a lexical scanner, 
which takes the sequence of words of a syntactically correct Python program, and transforms it
into a sequence of tokens. \(\tau_{4}^{-1}\) converts this token sequence into a parse tree,
according to Python's full grammar specification \cite{PythonFullGrammar}. \(\tau_{3}^{-1}\) 
transforms the parse tree into a more compact and easier-to-handle abstract syntax tree (AST),
according to Python's abstract syntax grammar specification \cite{PythonAbstractGrammar}.
So far, this process is equivalent to the functionality of CPython's parser module
\cite{CPythonCompiler}. \(\tau_{2}^{-1}\) converts the Python AST to an isomorphic AST
with a different underlying implementation. Finally, \(\tau_{1}^{-1}\) outputs a sequence
of tree-constructing actions that correspond to the leftmost generative story
of the previous component's AST.

\textbf{\textsc{Invertibility}}\ \ \ It is important to note that the above described transitions
are usually not reversible, certain restrictions have to be put in place to ensure one-to-one
correspondence. Having said that, the bijectivity of these functions is accepted to be given,
and the specifics of how it is achieved is not discussed in detail here.

\subsubsection{ASDL}

The AST itself adheres to Python's abstract syntax grammar, which is specified using
ASDL, the Abstract Syntax Description Language \cite{WangAppelKornSerra1997}.
A simplified version of ASDL is defined by the following EBNF \cite{Wirth1977} description:

\begin{verbatim}
    grammar        = types;
    types          = { type };
    type           = name, "=", constructors;
    constructors   = constructor, { "|", constructor };
    constructor    = name, [ fields ];
    fields         = "(", field, { ",", field }, ")";
    field          = name, [ "?" | "*" ], [ name ];
    letter         = "a" | ... | "Z";
    alpha_num      = "_" | letter | "0" | ... | "9";
    name           = letter, { alpha_num };
\end{verbatim}

\noindent
Let's introduce the following predicates to capture the semantics of ASDL:
 \begin{equation}
 \begin{split}
    Type(t) &= \text{"t is a type"} \\
    Prim(t) &= \text{"t is a primitive type"} \\
    Comp(t) &= \text{"t is a composite type"} \\
    Ctor(c) &= \text{"c is a constructor"} \\
    Field(f) &= \text{"f is a field"} \\
    SinField(f) &= \text{"f is a field with single cardinality"} \\
    OptField(f) &= \text{"f is a field with optional cardinality"} \\
    SeqField(f) &= \text{"f is a field with sequential cardinality"} \\
    CtorOf(c, t) &= \text{"c is a constructor of t"} \\
    FieldOf(f, c) &= \text{"f is a field of c"} \\
    TypeOf(t, f) &= \text{"t is the type of f"}
 \end{split}
 \end{equation}

\noindent
The following formulae state some of the basic properties of ASDL:
 \begin{equation}
 \begin{split}
    \forall t(Type(t) &\iff Prim(t) \oplus Comp(t)) \\
    \forall t(Prim(t) &\Rightarrow \nexists c(CtorOf(c, t))) \\
    \forall t(Comp(t) &\Rightarrow \exists c(CtorOf(c, t))) \\
    \forall f(Field(f) &\iff OneHot(SinField(f),\\ 
    & OptField(f), SeqField(f))) \footnotemark \\
    \forall c \forall t(CtorOf(c, t) &\Rightarrow Ctor(c) \land Type(t)) \\
    \forall f \forall c(FieldOf(f, c) &\Rightarrow Field(f) \land Ctor(c)) \\
    \forall t \forall f(TypeOf(t, f) &\Rightarrow Type(t) \land Field(f)) \\
    \forall f(&\exists! t(TypeOf(t, f)))\\
 \end{split}
 \end{equation}
\footnotetext{\(OneHot(p, q, r) := (p \land \lnot q \land \lnot r) 
\lor (\lnot p \land q \land \lnot r) \lor (\lnot p \land \lnot q \land r)\)}

\noindent
An ASDL description's semantics can be regarded as a sequence of formulae that
(1) create types, constructors and fields (2) determine field cardinalities 
(3) establish \(CtorOf\), \(FieldOf\) and \(TypeOf\) relationships.

\subsubsection{ASDL-based Abstract Syntax Trees}

An ASDL description defines a grammar. The sentences of a grammar defined by
an ASDL description can be transformed from and to ASTs, as is done by
\(\tau_{2}\) and \(\tau_{3}\). These ASTs build upon the constructs of the
ASDL description.

\noindent
Let's introduce the following new predicates:

\begin{align*}
RootType(t)   &= \text{"t is the root type"} \\
ValueOf(x, f) &= \text{"x is a value of f"} \\
TokenOf(x, t) &= \text{"x is a token of t"}
\end{align*}

\noindent
The following formulae state the properties of these newly introduced constructs:

\begingroup
\allowdisplaybreaks
\vspace{-.5cm}
\begin{gather*}
\forall t(RootType(t) \Rightarrow Type(t)) \\
\exists! t(RootType(t)) \\
\forall t(Prim(t) \Rightarrow \exists x(TokenOf(x, t))) \\
\forall t(Comp(t) \Rightarrow \nexists x(TokenOf(x, t))) \\
\forall f(SinField(f) \Rightarrow \exists! x(ValueOf(x, f))) \\
\forall f(OptField(f) \Rightarrow \forall x \forall y(ValueOf(x, f) 
\land ValueOf(y, f) \Rightarrow x = y)) \\
\forall x \forall f(ValueOf(x, f) \Rightarrow Field(f)) \\
\forall x \forall f \forall t(ValueOf(x, f) \land
TypeOf(t, f) \land Comp(t) \Rightarrow Ctor(x) \land CtorOf(c, t)) \\
\forall x \forall f \forall t(ValueOf(x, f) \land TypeOf(t, f)
\land Prim(t) \Rightarrow TokenOf(x, t)) \\
\forall x \forall t(TokenOf(x, t) \Rightarrow Type(t))
\end{gather*}
\endgroup

\noindent
It can be seen that the ASDL description defines exactly when a constructor can be
the value of a field, however, it doesn't state anything about the tokens corresponding
to a primitive type. Consider that information implicitly given for any ASDL
description, for example by an external source. The root type of an ASDL
description is the first type defined as part of it.

\textbf{\textsc{ASDL-based AST}}\ \ \ An ASDL-based abstract syntax tree is a
directed tree graph with a single root node. Its inner vertices are constructors,
while its leaves are either fieldless constructors or tokens. The root node is a
constructor of the grammar's root type. Each inner node has a number of fields,
and each of its fields has a number of values (based on their cardinality).
These values are edges pointing to a constructor node or a token node,
matching the field's type.

\textbf{\textsc{Completeness}}\ \ \ Such an AST is completed if all of its fields
are completed. Single fields are completed if they have exactly one value,
optional fields are completed if they have one value or they have no value
and have been explicitly flagged as completed, and sequential fields are
completed if they have been flagged as completed.

\subsubsection{ASDL-based AST Constructing Actions}

\(\tau_{1}\) defines a mapping between the ASDL-based ASTs - introduced in the
previous section - and a sequence of tree constructing actions. The set of
these action sequences serves as the intermediate meaning representation
\(IMR\) for solving the problem. The following tree constructing actions
are defined on a given ASDL-based AST \(\alpha\):

\textbf{\textsc{ApplyCtor}}\ \ \ An \(\textsc{ApplyCtor}(c)\) action takes
a constructor \(c\) defined in the ASDL grammar and appends a corresponding
new node to \(\alpha\)'s leftmost incomplete field. The precondition of this
action is the existence of an incomplete field in \(\alpha\), the leftmost
of which is \(f\), and that \(c\) matches the type of \(f\).

\textbf{\textsc{GenToken}}\ \ \ A \(\textsc{GenToken}(s)\) action has the
same effect and precondition as \textsc{ApplyCtor}, but instead of a
constructor, it appends a token.

\textbf{\textsc{Reduce}}\ \ \ A \(\textsc{Reduce}()\) action explicitly
flags \(\alpha\)'s leftmost incomplete field as complete. The precondition
of this action is the existence of an incomplete field in \(\alpha\),
the leftmost of which is \(f\), and that \(f\) either has a sequential
cardinality or an optional cardinality and no values.

Due to the transition system, action sequences that generate a completed
AST have a one-to-one correspondence with the elements of \(FMR\),
and the ones that generate an incomplete AST have a one-to-one correspondence
with the elements of \(V \setminus FMR\). This property is utilized by the
heuristic solution algorithm that solves the previously defined path-search
formulation of the intent parsing problem.

\subsection{Neural Network}

Heuristics can be used for many purposes in path-search algorithms. In the current
case, the heuristic used is a neural network whose predictions serve the two
requirements of the beam search algorithm: determine when a solution has been
reached, and label the out-edges of a vertex in proportion to how promising
following that edge seems to be in order to find a solution. Neither of these
objectives have a straightforward algorithmic solution, so it is better
to use a probabilistic approach, in the form of a neural network. In our case, 
this is the role of TranX. The following sections describe the more important
parts of it.

\subsubsection{Encoder}

Provided with the input natural language query \(q\) decomposed as a sequence
of words \(q = w_{1}...w_{|q|}\), the first stage of the neural network is an
encoder which encodes \(q\)'s information in a hidden state vector \(\mathbf{h} =
\{\mathbf{h}_{i}\}_{i=1..|q|}\), where \(\mathbf{h}_{i} \in \mathbb{R}\)
corresponds to the \(i\)-th input word \(w_{i}\) in \(q\).

The encoding is produced by a bidirectional LSTM \cite{HochreiterSchmidhuber1997,
SchusterPaliwal1997}, which goes over \(q\) first in a forward, then in a backward
order. The results are the partial hidden state components
\(\overrightarrow{\mathbf{h}}_{i} = \overrightarrow{f_{LSTM}}(\mathbf{w}_{i}, \overrightarrow{\mathbf{h}}_{i-1})\) and \(\overleftarrow{\mathbf{h}}_{i} =
\overleftarrow{f_{LSTM}}(\mathbf{w}_{i}, \overleftarrow{\mathbf{h}}_{i-1})\).
\(\mathbf{w}_{i}\) denotes the embedding of the \(i\)-th word \(w_{i}\), 
which is the index of the word in the vocabulary, or if the word is not
in the vocabulary, the index representing unknowns (\(<unk>\)).
The \(i\)-th component of the final hidden state is the concatenation of
these two partial hidden state components: \(\mathbf{h}_{i} = 
[\overrightarrow{\mathbf{h}}_{i} : \overleftarrow{\mathbf{h}}_{i}]\).

\subsubsection{Decoder - Hidden states}

The decoder component of the neural network takes the hidden state vector
\(\mathbf{h}\) produced by the encoder, as well as the input query \(q\),
and predicts the probability of an action \(a_{t}\) being correct.

This probability is conditional to the previous actions applied up to that
point. This is reflected by the decoder by maintaining an internal hidden
state, \(s_{t}\) at each time step that is calculated using the previous actions.
This \(s_{t}\) will be used when computing the action probabilities.
The exact formula is as follows:

\[\textbf{s}_{t} = f_{LSTM}([\textbf{a}_{t-1} : \tilde{\textbf{s}}_{t-1}], \textbf{s}_{t-1})\]

\textbf{\textsc{Action Embedding}}\ \ \ \(\textbf{a}_{t-1}\) is the embedding
of the previous action. Two embedding matrices are maintained, \(\textbf{W}_{R}\)
and \(\textbf{W}_{G}\). Each row in \(\textbf{W}_{R}\), except for the last one
is an embedding vector for an \textsc{ApplyCtor} action. The last row stands for
the \textsc{Reduce} action, hence it is handled as if it was a special \textsc{ApplyCtor}
action. \(\textbf{W}_{G}\) contains rows for the token generation actions. Actions of the
same type are differentiated by their parameter constructors/tokens, so as many
\textsc{ApplyCtor} actions exist as many constructors there are. The \textsc{GenToken}
actions have parameters that correspond to the words in the vocabulary as well
as the input sequence's tokens.

\textbf{\textsc{Attentional Vector}}\ \ \ \(\tilde{s_{t}}\) is the attention vector
calculated by \(\tilde{s_{t}} = \text{tanh}(\textbf{W}_{c}[\textbf{c}_{t} :
\textbf{s}_{t}])\). This vector has been introduced in \cite{LuongPhamManning2015}. % as Eq. (5)
\(\textbf{W}_{c}\) is the weight matrix of the concatenation layer used to
merge the context vector \(\textbf{c}_{t}\) and the decoder hidden
state \(\textbf{s}_{t}\).

\textbf{\textsc{Context Vector}}\ \ \ The context vector \(\textbf{c}_{t}\) is
computed by the annotations produced by the encoder (its hidden state vector
\(\textbf{h}\)), as described by \cite{BahdanauChoBengio2014}. The exact computation
is a weighted sum of the annotations:

\[\textbf{c}_{t} = \sum_{i=1}^{|q|}\alpha_{t_{i}}\textbf{h}_{i}\]

Since the context vector is a sum of all the input encoding, this attention
mechanism is global in nature. Here the attention weights \(\alpha_{t_{i}}\)
are calculated using:

\[\alpha_{t_{i}} = \frac{\exp(e_{t_{i}})}{\sum_{j=1}^{|q|}\exp(e_{t_{j}})}\]

The energy \(e_{t_{i}}\) of the weight \(\alpha_{t_{i}}\) is produced by an alignment
model and it acts as a score to show how important \(\textbf{h}_{i}\) is for the
calculation of the next decoder hidden state \(\textbf{s}_{t}\). The alignment
model utilized is the dot product score function introduced in \cite{LuongPhamManning2015}.
Its inputs are the encoding \(\textbf{h}_{i}\) and the previous decoder
hidden state \(\textbf{s}_{t-1}\):

\[e_{t_{i}} = \text{align}(\textbf{s}_{t-1}, \textbf{h}_{i}) = \textbf{s}_{t-1}^{T}\textbf{h}_{i}\]

\subsubsection{Decoder - Action probabilities}

The calculation of the probability \(p(A_{t} | \cap_{i=1}^{t-1} A_{i} \cap Q)\)
is identical to the method described in \cite{YinNeubig2018}. This process
depends on what kind of action \(A_{t}\) is. The \(\textsc{ApplyCtor}\) and
\(\textsc{Reduce}\) actions are calculated similarly, as the latter category
is handled as a special version of the former.

\textbf{\textsc{ApplyCtor Action}}\ \ \ The probability of applying a rule
\(r\) (which could be the reduce "rule") at time step \(t\) conditional
on the input query \(Q\) and the preceeding actions \(A_{i} \ (i \in [1,t-1])\) is:

\[p = \text{softmax}(\textbf{a}_{r}^{T}\textbf{W}\tilde{\textbf{s}}_{t} + \textbf{b})\]

\textbf{\textsc{GenToken Action}}\ \ \ This probability is the sum of the
probability of two cases: the case of generating a token from the vocabulary
and the case of copying a token from the input query. The following events are
introduced: \(GEN\) if \(A_{t}\) is a generation action, \(GEN_{v}\) if \(A_{t}\)
generates the word \(v\) from the vocabulary, \(COPY\) if \(A_{t}\) is a copy
action, \(COPY_{w}\) if \(A_{t}\) copies the word \(w\) from the input query.

\[p = p(GEN)p(GEN_{v}|GEN) + p(COPY)p(COPY_{w}|COPY)\]

\(p(GEN)\) and \(p(COPY)\) are based on the attention over the encodings by
a linear layer as in \(\text{softmax}(\textbf{W}\tilde{\textbf{s}}_{t})\)
(the weights are different for \(GEN\) and \(COPY\)). \(p(GEN_{v}|GEN)\) is
given by \(\text{softmax}(\textbf{a}_{v}^{T}\textbf{W}\tilde{\textbf{s}}_{t}
+ \textbf{b})\) whereas \(p(COPY_{w}|COPY)\) is predicted by a pointer network
\cite{VinyalsFortunatoJaitly2015} with a single linear layer.
The pointer network is utilized because the number of input tokens is variable.