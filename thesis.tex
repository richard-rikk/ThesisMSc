\documentclass[
	%parspace, % Térköz bekezdések közé / Add vertical space between paragraphs
	%noindent, % Bekezdésének első sora ne legyen behúzva / No indentation of first lines in each paragraph
	%nohyp, % Szavak sorvégi elválasztásának tiltása / No hyphenation of words
	%twoside, % Kétoldalas nyomtatás / Double sided format
	%draft, % Gyorsabb fordítás ábrák rajzolása nélkül / Quicker draft compilation without rendering images
	%final, % Teendők elrejtése / Set final to hide todos
]{elteikthesis}[2021/09/20]

% The minted package is also supported for source highlighting
\usepackage[newfloat]{minted}
\usepackage{booktabs}
\usepackage{cleveref}
\usepackage{enumitem}

% Dolgozat metaadatai
% Document's metadata
\title{Generating Algorithmic Patterns from Semi-structured Input Using a Transition-based Neural Network} % cím / title
\date{2022} % védés éve / year of defense

% Szerző metaadatai
% Author's metadata
\author{Richárd Rikk}
\degree{Artificial Intelligence MSc}

% Témavezető(k) metaadatai
% Superivsor(s)' metadata
\supervisor{Teréz A. Várkonyi} % belső témavezető neve / internal supervisor's name
\affiliation{Senior Lecturer} % belső témavezető beosztása / internal supervisor's affiliation
\extsupervisor{Balázs Pintér} % külső témavezető neve / external supervisor's name
\extaffiliation{Senior Lecturer} % külső témavezető beosztása / external supervisor's affiliation

% Egyetem metaadatai
% University's metadata
\university{Eötvös Loránd University} % egyetem neve / university's name
\faculty{Faculty of Informatics} % kar neve / faculty's name
\department{Dept. of Artificial Intelligence} % tanszék neve / department's name
\city{Budapest} % város / city
\logo{elte_cimer_szines} % logo

%%%%%%%%%%%%%%%  Add bibliography file %%%%%%%%%%%%%%% 
\addbibresource{cites.bib}

%%%%%%%%%%%%%%%  The document %%%%%%%%%%%%%%% 
\begin{document}

% Set document language
%\documentlang{magyar}
\documentlang{english}

% Teendők listája (final dokumentumban nincs)
% List of todos (not in the final document)
%\listoftodos[\todolabel]

%%%%%%%%%%%%%%% Title page (mandatory) %%%%%%%%%%%%%%%
\maketitle
\topicdeclaration{Program synthesis from natural language utterances has been an important goal of
artificial intelligence because of the many possibilities it offers to help
human-computer interactions. However, forming program code from natural language
is a challenging task because natural language utterances tend to be complex and
ambiguous with multiple possible interpretations.
 
The problem can be tackled in multiple ways. Some models do not constrain their
natural language inputs and try to learn to represent them despite the
complexity and ambiguities. Other models severely limit the space of the
possible inputs, gaining accuracy but losing generality.
 
In my thesis, I try to find a middle ground between these approaches. I apply a
context-free grammar to generate structured utterances that are close to natural
language. The resulting sentences are between constrained and unconstrained
inputs and gain many positive properties. From the structured side, one can gain
easier training and dataset generation, while from the natural language side
ease of use and interpretability. Our method allows for the generation of large
datasets where the inputs can be used with state of the art transition-based
models to achieve good results. The model trained on these semi-structured
inputs is able to generalize to natural language utterances.}

%%%%%%%%%%%%%%% Table of contents (mandatory) %%%%%%%%%%%%%%%
\tableofcontents
\cleardoublepage

%%%%%%%%%%%%%%% Main content %%%%%%%%%%%%%%%
\section{Motivation}
Synthesizing program code from natural language is a challenging task as natural
language utterances tend to be ambiguous and require substantial prior knowledge
to interpret. Recent solutions approach these difficulties in different ways. Some models do not
constrain their inputs and operate with a large variety of sentences, but tend
to be less accurate. Others limit the space of possible inputs, requiring them
to meet a fixed structure which makes them more similar to code than language.

This thesis offers a middle ground between these approaches. We train a
transition-based neural network on descriptions of programming tasks generated by
using context-free grammar and templates. We show that the model is able to
generalize and can solve synthesis problems described in natural language.
The approach could be very useful in a wide variety of program synthesis problems where natural language data is scarce.
\cleardoublepage

\section{Introduction}
%
Generating program code from natural language utterances plays a significant
role in Artificial Intelligence research \cite{useful1, useful2, useful3}. There are several reasons for that. The realization of such technology would allow professionals to produce their
software faster by cutting development time significantly. Also, it would be
possible for anyone without coding experience to create software and it could
potentially reduce the barrier of entry into computer science. However, creating
such technology is a challenging task due to the complexity and ambiguity of
natural language. 

Considering previous work, we can distinguish two main categories of program
synthesis methods with respect to the naturalness of the input. Some models do
not place any or significant constraints on the input utterances. Other models
demand that the inputs meet strict rules concerning their form. The two
categories of models face different challenges.

When using natural language utterances without any constraints\cite{text1,
  text2}, the model has to learn the complex relationship between language and
code, which requires powerful models and large datasets. Providing datasets with
adequate cardinality and quality for these models needs a lot of effort and
resources.

In contrast, models working with constrained, structured inputs tend to perform
better and are able to avoid some of the pitfalls of natural language. However,
the inputs are closer to formal than natural language, so we lose the
naturalness of the interface and the accessibility for people without coding
experience.

In this thesis, we combine the advantages of unconstrained and structured models.
We present a method that learns from structured inputs but synthesizes programs
from unconstrained natural language. We show that training the transition-based
\cite{trans} neural network TranX \cite{tranx} on \emph{(structured utterance, code
snippet)} pairs enables \emph{generalization} to unconstrained natural language
inputs, in case enough training data is given.

The structured training dataset is generated automatically, which allows us to obtain the large dataset necessary for generalization. In the dataset, structured utterances are created using templates and a context-free grammar (CFG). The code snippets are basic algorithmic patterns \cite{progT}. Basic algorithmic patterns aim to solve common programming problems and are constructed from elementary operations using formal mathematical methods that guarantee their correctness. 

The main advantage of our method is that our \emph{example generator} is capable
of generating datasets with an arbitrary number of training examples given a
suitable grammar and templates. These examples are then used to train the neural
network which can generalize to unconstrained inputs. The method is general, as
the templates, the grammar, and the code snippet generator can be tailored to
specific domains.

In the next sections, we go over our method in detail. First, an overview is
provided highlighting all the main parts of our model. Then, the algorithmic
patterns are discussed that are the basis of the generated code. Next is our
method for utterance generation including the structure of our grammar and the
idea of templates. After this, the role of TranX and its modifications are
explained. Finally, various datasets are introduced, and their creation,
cardinality and purpose are also described.

\section{Related Work}

In this section, methods dealing with program synthesis are going to be
discussed. As we remarked earlier, we distinguish two categories of program
synthesis methods with respect to the naturalness of the input. The first
category includes the usage of natural language utterances without any constraints. In
this category, e.g., TranX \cite{tranx}, Allamanis et al. \cite{project3}, and
Bunel et al. \cite{project4} can be mentioned.

TranX is transition-based neural semantic parser using the abstract syntax
description of the target programming language. The task of the neural network
is to learn the steps of creating an abstract syntax tree (AST) representation
for the input natural language utterance. The method is highly generalizable as
only the abstract syntax description has to be replaced to change the target
programming language. However, the transformation function that converts an AST
into a program code has to be defined by hand.

In \cite{project3}, authors map short natural languages to code snippets and
vice versa using a probabilistic model. They build up a directed parse tree
where the child nodes are generated based on the natural language input and the
partial tree.

In \cite{project4}, reinforcement learning is used on top of a supervised model
with the objective to maximize the likelihood of generating semantically correct
programs. More precisely, each input-output pair is embedded jointly by a
convolutional neural network. Then a long short-term memory decoder is used for
each example. The results of all decoders are maxpooled, then the reinforcement
model is used to get the prediction.

In the second category of program synthesis, where models work with constrained
inputs, Quirk et al. \cite{project1} and Yao et al. \cite{project11} are
interesting examples.

Quirk et al. in \cite{project1} show a way to map natural language “if-then” rules
to executable code. Their approach creates ASTs from structured utterances
with the help of a probabilistic log-linear text classifier. The ASTs represent
if-then recipes matching a formal grammar created by the authors.

Yao et al. in \cite{project11} use hierarchical reinforcement learning where the
agent is allowed – instead of synthesizing a program in one shot – to ask
follow-up questions from the user. This is done by defining different subtasks
like predicting trigger/action channel/function and using a model with a
hierarchical policy. A high-level policy decides the order of the subtasks to
work on, and low-level policies decide whether a clarifying question is needed
or the given subtask component can be predicted. The goal of the agent is to
maximize the parsing accuracy and minimize the number of questions needed.

Our solution sits on the middle ground between the unconstrained and constrained
approaches and tries to have the best of both worlds. The sentences become
interpretable just like natural utterances but they remain structured enough for
automated generation which facilitates the supervised learning of models.
\section{Overview}

\insertPic{method}{0.65}{Utterance generation from the handcrafted inputs. Given
  the grammar and natural language utterance examples, templates were
  constructed that mimic the structure of the examples but contain all necessary
  information to solve the given task. The templates are used to create
  (structured task description, Python code snippet) pairs by fixing the
  \texttt{<indicator>} to obtain the pattern specific templates and then
  randomly selecting values for all the other fields from lists of possible
  values.}

In this section, our training dataset generation method is presented.
The method can be seen in \cref{fig:method}. First, a context-free
grammar (CFG) is created -- in Backus-Naur Form (BNF)\cite{bnf} -- for
some set of tasks. The tasks, in our case are solvable by algorithmic
patterns and are expressed in natural language. After that, templates
are handcrafted so that they satisfy the CFG and they cover the
selected tasks.

The generator is using the templates to generate the \emph{(structured task
description, Python code snippet)} pairs in multiple steps. First, for a given
template, an \verb|<indicator>| value is selected. Second, all fields of this
template receive a concrete value so the input sentence is created. The last
step is to produce the code snippet for the generated sentence, which means
using the value of the \verb|<indicator>| field to define which algorithm should
be used and then substituting the rest of the values.

The CFG has been constructed so that it fits both the algorithmic patterns
discussed in \cite{progT} as well as the English language. These patterns allow
us to define a broad set of tasks, which have well defined solutions. Using
algorithmic patterns has the advantage that they are simple and mathematically
sound.

After generating the dataset containing enough \emph{(structured task
description, Python code snippet)} pairs, the transition-based neural
network TranX \cite{tranx} is trained on it.
Even though it is trained on structured inputs, we test its ability
to map \emph{unconstrained} task descriptions to code snippets.

TranX is usually used to map unconstrained natural language utterances to formal
meaning representations, which are typically code snippets. It has already
achieved good results on many datasets, targeting different programming
languages. In this thesis, we focus on Python 3 for its simplicity and ease of
use. It is important to highlight that our grammar is language-independent and
so is TranX:~it does not assume constraints on the input utterance or that it is
structured in any way. We aim to show that structuring the input sentences
allows TranX to generalize to natural language utterances.

Our templates have a bridging role. Their purpose is to create a more structured
representation by using natural language task descriptions as an example, while
maintaining their meaning and if necessary, extending them to make them more
interpretable. This is needed because these descriptions are ambiguous. The
ambiguity can be reduced by simplifying certain expressions in the sentences.
However, this has to be done sparingly, otherwise the result becomes code-like,
which would alienate people without coding experience.
\section{Algorithmic Patterns}
\label{sec:pattern}

Generic algorithmic patterns\cite{progT, progT1} are applied to
solve frequently occurring programming problems, like counting
or summing elements of a collection\cite{progSolve}. The problems can appear in different forms
but their substance is the same, so we can use analogies with the algorithmic
patterns to solve them. This section describes the chosen algorithmic patterns
and their corresponding Python snippets.

Algorithmic patterns come from the area of analogous programming\cite{progT}, a
formal software development method where programs are composed from provably
correct parts.

We are using the following algorithmic patterns, detailed below:

\begin{enumerate}[noitemsep]
  \item Summation
  \item Count
  \item Maximum search
  \item Selection
  \item Decision
  \item Assortment
\end{enumerate}

\subsection{Formalism}\label{sec:def}

In order to discuss the used patterns effectively, formalisms need to be
introduced. These will be used to define algorithmic patterns and they are
relevant in the construction of the CFG rules.

Let us introduce the set \( \mathcal{H} \), a set on which a given task is
defined. For example, if our task is to sum the elements of set
\( \{ 1, 2, 6, 7, 12, 5 \} \), then \( \mathcal{H} = \mathbb{N} \). In this case
\( h \in \mathcal{H} \) is a natural number.

In our tasks the elements are members of a \( collection \) which is a sequence
holding the relevant elements of \( \mathcal{H} \). We can iterate over a
collection. We are using Python 3, so in our concrete case \( collection \) is a
Python list and \( h \in collection \) is an element of the list.

An important thing to consider is that in many cases, \( \mathcal{H} \) contains
compound elements. For example, if our task is to find the tallest student in a
classroom, \( \mathcal{H} \) can contain objects (records) of students with
different attributes. We have to select the relevant property from e.g., name,
height, grades, etc. We denote the set of values of the relevant property by
\( \mathcal{V} \). In the example height is the relevant property, therefore
\( \mathcal{V} = \mathbb{N} \).

We also introduce a mapping function \( f: \mathcal{H} \to \mathcal{V} \) to map
the elements of the \( collection \) to their relevant value. In the example
\( f \) would map students to their heights. In case of primitive types, like
natural numbers, \( \mathcal{H} = \mathcal{V} \) and \( f \) is the identity
function.

Finally, the filtering function \( c: \mathcal{H} \to \mathbb{L}\) maps
\( x \in \mathcal{H} \) to a logical value, based on a condition. The role of
function \( c \) is to determine which elements of the \( collection \) take
part in the calculation.

\subsection{The Algorithmic Patterns}

This section describes the algorithmic patterns we use and provides a generic
Python code snippet for each. Since our dataset contains \emph{(utterance, code)}
pairs, these snippets can be used as templates for the latter. During the
creation of these snippet templates the emphasis was on understandability,
therefore it is possible that some of these templates are not the most
efficient. However changing the templates are relatively easy if efficiency
becomes more important later.

\paragraph{Summation}\label{sec:sum}
is one of the most applicable patterns\cite{Sum}. It is used when items of a
collection need to be summed:
\begin{listing}[H]
\begin{minted}{python}
    r = sum(f(x) for x in collection if c(x) )
\end{minted}
\end{listing}


\paragraph{Count}
is used to count those items of a collection that satisfy a certain condition:
\begin{listing}[H]
\begin{minted}{python}
    r = sum(1 for x in collection if c(x) )
\end{minted}
\end{listing}

\paragraph{Maximum search}
is an algorithmic pattern where our task is to find the maximum element
according to comparison operator \( \leq \):
\begin{listing}[H]
\begin{minted}{python}
    r = max(f(x) for x in collection),
\end{minted}
\end{listing}

where the \( max \) function uses the comparison operator. Although this
formulation describes the algorithm well when the task is to find the maximal
element according to the operator \( \leq \), it falls short when a condition is
introduced. In order to handle such cases, one can extend the algorithm with a
condition. In this case, it is assumed that there is at least one
element satisfying the condition. The new form of the algorithm augmented by
condition \( c \):

\begin{listing}[H]
\begin{minted}{python}
    r = max(f(x) for x in collection if c(x) )
\end{minted}
\end{listing}

\paragraph{Selection}
is used when the task is to find the first element matching a condition. The
precondition is that at least one element satisfies the condition.
\begin{listing}[H]
\begin{minted}{python}
    r = next(f(x) for x in collection if c(x) )
\end{minted}
\end{listing}


\paragraph{Decision}
 determines whether there is an element that satisfies the given condition \( c \). 
\begin{listing}[H]
\begin{minted}{python}
    r = any(c(x) for x in collection)
\end{minted}
\end{listing}

The problem to decide whether all elements meet the given condition \( c \) can
be formulated in similar fashion.

\begin{listing}[H]
\begin{minted}{python}
    r = all(c(x) for x in collection)
\end{minted}
\end{listing}

\paragraph{Assortment}
is used when a set of elements needs to be separated into two subsets: one where
all elements meet a certain condition and another where elements do not meet
this condition.
\begin{listing}[H]
\begin{minted}{python}
    r1 = [x for x in collection if c(x) ]
    r2 = [x for x in collection if not c(x)],
\end{minted}
\end{listing}
where \( r1 \) and \( r2 \) do not contain any identical items.
\section{Utterance Generation}\label{sec:STDT}

This section describes our method to generate training examples. Our goal is to
create sentences that are structured but similar to unstructured natural
language. The structure is given by the CFG and the templates crafted carefully
so the resulting sentences still satisfy the syntactic rules of natural
language. We call these sentences structured task descriptions. Each of these
descriptions has a natural language equivalent. \cref{fig:natural} shows a
(structured task description, natural language sentence) pair. The task
description will be used as the running example in this section.

\insertPic{natural}{0.75}{A structured task description (bottom) and its natural
  language counterpart (top). The intermediate step is included for explanation
  purposes, in order to make the same parts in the two sentences more
  identifiable.}

The templates are handcrafted by expanding the algorithm rule (\cref{fig:arule})
with extra tokens based on natural language utterances describing tasks.
Templates are created for specific algorithmic patterns in order to get a result
close to natural language. This is why in each template the \verb|<indicator>|
field is fixed -- its value is predetermined by the indicator rule
(\cref{fig:irule}). \cref{fig:template} shows a valid template used during the
dataset generation.

\insertPic{arule}{0.75}{The algorithm rule. The templates are created by
  expanding this rule with additional tokens to make it similar to natural
  language.}

\insertPic{irule}{0.75}{The indicator rule. Each possible value maps to one of
  the algorithmic patterns described in \cref{sec:pattern}.}

\insertPic{template}{0.64}{A template to generate structured task descriptions
  of the summation algorithmic pattern. It has the same fields as the
  nonterminals of the algorithm rule in \cref{fig:arule} as it was created by
  expanding that rule with additional tokens (except that it has a single
  condition).}

The templates give structure to the task descriptions. Each field in a template
can be filled in multiple ways (\cref{fig:task}). We can obtain many examples
from a template so we can compile a diverse dataset by using a relatively small
number of templates. The \texttt{<conditions>} rule shown in \cref{fig:crule}
also helps to make our dataset more diverse as it can generate a wide variety of
conditions.

\insertPic{task}{0.59}{Creating a structured task description from a template by
  filling in its fields. The value of each field is randomly selected from
  its list of possible values. The figure has two possible values in each list
  for demonstration.}

\insertPic{crule}{0.75}{The condition rule can generate different conditions in
  the templates.}

The Python snippets are generated from the filled templates. First the
algorithmic pattern and its code structure is selected based on the
\texttt{<indicator>} field. Then the rest of the fields are used to substitute
the corresponding components in the code structure to yield the code snippet.
\cref{fig:connect} shows an example of code snippet generation from a template.

\insertPic{connect}{0.6}{Given a filled template, one can produce the
  corresponding code snippet by selecting the code structure based on the
  \texttt{<indicator>} field and filling its components.}
\section{Transition-based Neural Network}

As the model we use TranX\cite{tranx}, an abstract syntax network
\cite{abstractNetworks} using a global attention mechanism \cite{translation,
  LuongPhamManning2015} and an additional pointer network
\cite{VinyalsFortunatoJaitly2015}.

The core of TranX is a transition system which maps natural language utterances
to abstract syntax trees (ASTs) using a series of tree-constructing actions. The
ASTs are used as domain-independent intermediate meaning representations, and
are converted later into domain-specific meaning representations like the lambda
calculus or Python programs. The ASDL framework \cite{asdl} is used to define
the ASTs. The tree-constructing actions apply the constructors of the ASDL
grammatical formalism to build the AST in a top-down, left-to-right order.

The network learns in a supervised fashion, using (utterance, AST) pairs. The
ASTs are created from Python code by the reverse of the generation process: the
sequence of source code tokens is transformed into a Python AST, the Python AST
is converted into a domain-independent isomorphic AST, from which finally the
sequence of tree-constructing actions is obtained.

We had to apply some minor modifications to the TranX software \cite{tranxRepo}
to make it scale up to our experiments. The pickling\cite{oliphant2007python}
process and the training loop were changed to support arbitrary large datasets.
Other than these two changes the software is the same as the original found on
GitHub \cite{tranxRepo}. The architecture is unaffected by our modifications and
we use the same hyperparameters as \cite{tranx}. A complete list of
hyperparameters can be found in Appendix~\ref{sec:hype}.
\section{Evaluation}

In this section, we discuss the datasets created and the results obtained. We
have two datasets: one that consists of the (structured task description, Python
code snippet) pairs for training, and one that has unstructured natural language
task descriptions for testing.

\subsection{Datasets}

This section describes the two datasets introduced above. In order to test how
the model generalizes to unconstrained natural language, the datasets have been
created with different methods and priorities in mind.

The first dataset is used for training TranX and is generated by the process
described in \cref{sec:STDT}. This dataset has been created by using templates so
the emphasis is on the structure over anything else. Such generated utterances
do not have to be perfectly correct either syntactically or semantically. The
template fields have to be easily recognizable which is required to produce the
code snippets. The dataset contains \( 42000 \) examples: \( 6 \) algorithmic
patterns each with \( 7 \) templates and each template has \( 1000 \) examples.

The second dataset, which is handmade, contains unconstrained natural language
task descriptions and is used for testing. This dataset prioritizes semantic and
syntactic correctness. Still, all information is included in the sentences that
is needed to solve the task (i.e., create the Python code snippet) and if
possible the order of relevant tokens matches the algorithm rule. This dataset
contains \( 60 \) examples, 10 examples each for the six algorithmic patterns.

\subsection{Results}

In this section, the results are presented. The Corpus BLEU\cite{bleu} metric is
used, a standard metric when it comes to comparing sequences of strings. The
authors of TranX also used this metric to measure its performance on Python 3
output~\cite{tranxRepo}. We compare our results with their results of TranX on
the Conala\cite{conala} dataset. This dataset does not use any predefined
structure.

First, we tested TranX on the generated dataset it has been trained on. It has
achieved very good results with a Corpus BLEU score of 0.98. This is not
surprising since our templates have been relatively simple and due to the nature
of the generated dataset, the grammatical structure is often repeated, which
makes it easier to distinguish the template fields from the extra tokens.

Achieving high scores on our generated dataset is certainly good, but the real
test is how the model performs on more realistic sentences. In order to test
that, we used the second, handmade dataset, which has semantically and
syntactically correct natural language sentences while it also contains all the
information needed to synthesize the corresponding programs.

\cref{tab:result} shows that TranX was able to generalize to natural language
task descriptions despite the fact that it is trained on inputs generated by
templates. It achieved a relatively good BLEU score compared to the results of
the Conala dataset which does not contain any predefined structure.

\begin{table}[h]
  \normalsize \centering
  \begin{tabular}{lrl}
    \toprule
    Dataset  & Model & Metric \\
    \midrule
    Generated &  0.98  & Corpus BLEU \\
    Handmade  &  0.43  & Corpus BLEU \\
    Conala    &  0.245 & Corpus BLEU \\
    \bottomrule
  \end{tabular}
  \vspace{0.1cm}
  \caption{Model Corpus BLEU score on different datasets.}
  \label{tab:result}
\end{table}

\subsection{Discussion}

Getting good results on the generated dataset has been expected because of the
fixed structure of our training input sentences. Templates have ensured that the
important tokens are in predictable positions. Adding more templates to each
algorithm has diversified our dataset and has allowed the nonterminals of the
algorithm rule to appear in different locations in the structured task
descriptions. Despite this, overfitting on sentences created by templates is a
real possibility and has to be avoided if we want to process natural task
descriptions.

The handmade dataset has been created to test if TranX has learned to generalize
from structured to natural language task descriptions. TranX has been able to
reach considerably better results on this handmade dataset than on Conala. This
means that the model must be capable of some level of generalization and
recognize our CFG-grammar in a natural language environment.

It is also clear that TranX has performed worse on the handmade dataset compared
to the generated dataset. However, the handmade dataset contains more complex
examples with structures that TranX has not seen during training.

It is very likely that our results have been caused by two major factors. First,
the fixed structure of our inputs have made our utterances more regular hence
more recognizable for TranX. The second factor could have been the cardinality
of our dataset. Since by using our method it is easy to generate large datasets,
neural networks can work with more data and see more examples.
\section{Conclusion}

Program synthesis from natural languages is a challenging task. Some methods
restrict possible inputs by having them conform to a specific structure while
others do not. Our method combines the advantages of these two approaches.

We proposed a method that can generate a virtually unlimited number of
structured training examples. We also demonstrated that training the transition
neural network TranX \cite{tranx} using these structured examples allows it to
generalize to unconstrained natural language uttrances. Taken together we
constructed a method to train program synthesis models in domains where natural
language data is scarce or nonexistent.

An interesting future research direction would be to investigate whether the
grammar and the templates themselves can be inferred from a natural language
corpus. If we could infer these then the whole process could be automatized: we
could automatically augment natural language corpora with lots of synthetic
examples to train models that generalize to natural language.
\section*{Acknowledgements}

This material is based upon a research accomplished in the framework of
EFOP-3.6.3-VEKOP-16-2017-00001: Talent Management in Autonomous Vehicle Control
Technologies. The Project is supported by the Hungarian Government and
co-financed by the European Social Fund. We would like to thank Tamás Dina for
the helpful discussions.



\cleardoublepage

% Függelékek (opcionális) - hosszabb részletező táblázatok, sok és/vagy nagy kép esetén hasznos
% Appendices (optional) - useful for detailed information in long tables, many and/or large figures, etc.
\appendix
\section{Our Context-Free Grammar}\label{app:bnf}

Our context-free grammar is defined in BNF. The rules are the following:
\begingroup \fontsize{9pt}{9pt}\selectfont

    \begin{verbatim}
    <root>       ::= <algorithm>
    <algorithm>  ::= <indicator>[<item_property>]<item>[<conditions>]
                    <collection>
    <indicator>  ::= "Sum" | "Decide" | "Select" | "Search" | "Count" |
                     "Maximum" | "Assort"
    <item_property> ::= <characters> <conjunctive>
    <conjunctive>   ::= "of"
    <item>          ::= <characters>
    <conditions>    ::= <condition><cc>[<conditions>]
    <cc>            ::= "and" | "or" | "not" | "and not" |
                                 "or not" | ""
    <condition> ::= <identifier> <relation> <rhs>
    <relation>  ::= <equality> | <comparison>
    <equality>  ::= <equal> | <nequal>
    <equal>  ::= "is" | "equals" | "=" | "==" | "is equal to"
    <nequal> ::= "is not" | "isn’t" | "!=" | "<>" | "is not equal to"
                 "not equals"
    <comparison> ::= { <comp_less> | <comp_more> | <comp_less_e> |
                       <comp_more_e> } [<than>]
    <than>      ::= "than"
    <comp_less> ::= ["is"]{"less" | "smaller" | "lower" | "<" }
    <comp_more> ::= ["is"]{"more"| "bigger" | "larger" | "greater" |
                    ">"}
    <comp_less_e> ::= ["is"]{"less than or equal to" | "<=" | "at most"}
    <comp_more_e> ::= ["is"]{"more than or equal to" | ">=" | "at least"}
    <collection>  ::= ["in"] <container>
    <container>   ::= <characters>
    <rhs>         ::= <characters>
    <algorithm_conjunction> ::= "within" | "after" | ""
    <identifier> ::= <item> | <item_property>
    <characters> ::= <characters><character> | <character>
    <character>  ::= "a".."z" | "A".."Z" | "0".."9"  
    \end{verbatim}  
\endgroup \cleardoublepage
\section{Hyperparameters}\label{sec:hype}

TranX has been trained with the following hyperparameters.

\begin{table}[h]
  \normalsize \centering
  \begin{tabular}{lr}
    \toprule
    Hyperparameter  & Value \\
    \midrule
    dropout             &  0.30\\
    hidden\_size        &  256 \\
    embed\_size         &  128 \\
    action\_embed\_size &  128 \\
    field\_embed\_size  &  64  \\
    type\_embed\_size   &  64  \\
    ptrnet\_hidden\_dim &  32  \\
    lr                  &  0.001\\
    beam\_size          &  15  \\
    max\_epoch          &  2  \\
    batch\_size         &  128 \\
    \bottomrule
  \end{tabular}
  \vspace{0.1cm}
  \caption{The hyperparameters of TranX.}
  \label{tab:hype}
\end{table}
\section{The Templates Used for Dataset Generation}

In this appendix we provide a list of all the templates used for dataset
generation. The templates are separated by an empty line. The last line of each
template definition describes the structure of the condition. If the last line
is \verb"none" then there is no condition for the given template.

\begingroup \fontsize{9pt}{7pt}\selectfont
    \begin{verbatim}
    What is the <indicator> of <item> in <collection> ?
    none

    What is the <indicator> of the <item_property> of <item>
    where <conditions> in <collection> ?
    <identifier> <relation> <rhs>

    What is the <indicator> of the <item_property> of <item>
    where <conditions> in <collection> ?
    <item_property> <relation> <rhs>
    
    What is the <indicator> of the <item_property> of <item>
    where <conditions> in <collection> ?
    <identifier> <relation> <rhs> <cc> <item> <relation> <rhs>

    <indicator> of <item> in <collection> !
    none

    <indicator> of the <item_property> of <item>  
    where <conditions> in <collection> ?
    <identifier> <relation> <rhs>

    <indicator> of the <item_property> of <item>  
    where <conditions> in <collection> ?
    <identifier> <relation> <rhs> <cc> <item> <relation> <rhs>

    <indicator> if there is a <item_property> of a <item> 
    where <conditions> in <collection> ?
    <identifier> <relation> <rhs>

    <indicator> if there is a <item_property> of a <item> 
    where <conditions> in <collection> ?
    <item_property> <relation> <rhs>

    <indicator> if there is a <item_property> of a <item> 
    where <conditions> in <collection> ?
    <identifier> <relation> <rhs> <cc> <item> <relation> <rhs>

    <indicator> if everything is a <item_property> 
    of a <item> where <conditions>
    in <collection> ?
    <identifier> <relation> <rhs>

    <indicator> if everything is a <item_property> 
    of a <item> if <conditions> in <collection> ?
    <identifier> <relation> <rhs> <cc> <item> <relation> <rhs>

    <indicator> is a <item_property> of a <item> in <collection> 
    where <conditions>  ?
    <identifier> <relation> <rhs>

    <indicator> is a <item_property> of a <item> in <collection>
    where everything <conditions>  ?
    <identifier> <relation> <rhs>

    <indicator> a(n) <item_property> of <item> 
    where <conditions> in <collection> !
    <identifier> <relation> <rhs>
    
    <indicator> a(n) <item_property> of <item> 
    where <conditions> in <collection> !
    <item_property> <relation> <rhs>

    <indicator> a(n) <item_property> of <item> 
    where <conditions> in <collection> !
    <identifier> <relation> <rhs> <cc> <item> <relation> <rhs>

    <indicator> a(n)  <item>  if <conditions> in <collection> !
    <identifier> <relation> <rhs>

    <indicator> a(n)  <item>  where <conditions> in <collection> !
    <identifier> <relation> <rhs> <cc> <item> <relation> <rhs>

    <indicator> a(n)  <item> only if <conditions> in <collection> !
    <identifier> <relation> <rhs>

    <indicator> the  <item>  if <conditions> in <collection> !
    <identifier> <relation> <rhs> <cc> <item> <relation> <rhs>

    <indicator> the <item> in <collection> !
    none

    <indicator> the <item> where <conditions> in <collection> !
    <identifier> <relation> <rhs>

    <indicator> the <item> where <conditions> in <collection> !
    <identifier> <relation> <rhs> <cc> <item> <relation> <rhs>

    <indicator> the <item_property> of the <item>
    in <collection> !
    none

    <indicator> the <item_property> of the <item> 
    where <conditions>  in <collection> !
    <identifier> <relation> <rhs>

    <indicator> the <item_property> of the <item> 
    where <conditions>  in <collection> !
    <item_property> <relation> <rhs>

    <indicator> the <item_property> of the <item> 
    where <conditions>  in <collection> !
    <identifier> <relation> <rhs> <cc> <item> <relation> <rhs>

    What is the <indicator> <item> in <collection> ?
    none

    What is the <indicator> <item_property> of <item>
    in <collection> ?
    none

    <indicator> <item> in <collection> !
    none

    What is the <indicator> <item_property> of <item>
    in <collection> !
    none

    What is the <indicator> <item_property> of <item> 
    where <conditions> in <collection> ?
    <identifier> <relation> <rhs>

    What is the <indicator> <item_property> of <item> 
    where <conditions> in <collection> ?
    <item_property> <relation> <rhs>

    What is the <indicator> <item_property> of <item> 
    where <conditions> in <collection> ?
    <identifier> <relation> <rhs> <cc> <item> <relation> <rhs>

    <indicator> <item> where <conditions> in <collection> !
    <identifier> <relation> <rhs>

    <indicator> <item> where <conditions> in <collection> !
    <identifier> <relation> <rhs> <cc> <item> <relation> <rhs>

    <indicator> <item> where <conditions> in <collection> !
    <identifier> <relation> <rhs>

    <indicator> <item> where <conditions> in <collection> !
    <identifier> <relation> <rhs> <cc> <item> <relation> <rhs>

    <indicator> the <item_property> of <item> 
    where <conditions> in <collection> !
    <identifier> <relation> <rhs>

    <indicator> the <item_property> of <item> 
    where <conditions> in <collection> !
    <item_property> <relation> <rhs>

    <indicator> the <item_property> of <item> 
    where <conditions> in <collection> !
    <identifier> <relation> <rhs> <cc> <item> <relation> <rhs>
    \end{verbatim}
\endgroup
\cleardoublepage

% Irodalomjegyzék (kötelező)
% Bibliography (mandatory)
\phantomsection
\addcontentsline{toc}{section}{\biblabel}
\printbibliography[title=\biblabel]
\cleardoublepage

% Ábrajegyzék (opcionális) - 3-5 ábra fölött érdemes
% List of figures (optional) - useful over 3-5 figures
\phantomsection
\addcontentsline{toc}{section}{\lstfigurelabel}
\listoffigures
\cleardoublepage

% Táblázatjegyzék (opcionális) - 3-5 táblázat fölött érdemes
% List of tables (optional) - useful over 3-5 tables
\phantomsection
\addcontentsline{toc}{section}{\lsttablelabel}
\listoftables
\cleardoublepage

% Algorithmusjegyzék
% List of algorithms
% \phantomsection
% \addcontentsline{toc}{section}{\lstalgorithmlabel}
% \listofalgorithms
% \cleardoublepage

% Forráskódjegyzék (opcionális) - 3-5 kódpélda fölött érdemes
% List of codes (optional) - useful over 3-5 code samples
% \phantomsection
% \addcontentsline{toc}{section}{\lstcodelabel}
% \lstlistoflistings
% \cleardoublepage

% Jelölésjegyzék (opcionális)
% List of symbols (optional)
%\printnomenclature

\end{document}
